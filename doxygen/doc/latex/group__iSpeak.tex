\section{i\+Speak}
\label{group__iSpeak}\index{i\+Speak@{i\+Speak}}


Acquire sentences over a yarp port and then let the robot utter them, also controlling the facial expressions.  


Acquire sentences over a yarp port and then let the robot utter them, also controlling the facial expressions. 

\hypertarget{group__windows-tts_intro_sec}{}\subsection{Description}\label{group__windows-tts_intro_sec}
The behavior is pretty intuitive and does not need any further detail.~\newline
\hypertarget{group__windows-tts_lib_sec}{}\subsection{Libraries}\label{group__windows-tts_lib_sec}

\begin{DoxyItemize}
\item Y\+A\+R\+P libraries.
\item Packages for speech synthesis (e.\+g. festival, espeak, ...).
\end{DoxyItemize}\hypertarget{group__windows-tts_parameters_sec}{}\subsection{Parameters}\label{group__windows-tts_parameters_sec}
--name {\itshape name} 
\begin{DoxyItemize}
\item The parameter {\itshape name} identifies the unique stem-\/name used to open all relevant ports.
\end{DoxyItemize}

--robot {\itshape robot} 
\begin{DoxyItemize}
\item The parameter {\itshape robot} specifies the robot to connect to.
\end{DoxyItemize}

--period {\itshape T} 
\begin{DoxyItemize}
\item The period given in \mbox{[}ms\mbox{]} for controlling the mouth.
\end{DoxyItemize}

--package {\itshape pck} 
\begin{DoxyItemize}
\item The parameter {\itshape pck} specifies the package used for utterance; e.\+g. \char`\"{}festival\char`\"{}, \char`\"{}espeak\char`\"{}, ...
\end{DoxyItemize}

--package\+\_\+options {\itshape opt} 
\begin{DoxyItemize}
\item The parameter {\itshape opt} is a string specifying further command-\/line options to be used with the chosen package. Refer to the package documentation for the available options.
\end{DoxyItemize}\hypertarget{group__windows-tts_portsa_sec}{}\subsection{Ports Accessed}\label{group__windows-tts_portsa_sec}
At startup an attempt is made to connect to /$<$robot$>$/face/emotions/in port.\hypertarget{group__windows-tts_portsc_sec}{}\subsection{Ports Created}\label{group__windows-tts_portsc_sec}

\begin{DoxyItemize}
\item {\itshape /} $<$name$>$\+: this port receives the string for speech synthesis. In case a double is received in place of a string, then the mouth will be controlled without actually uttering any word; that double accounts for the uttering time. ~\newline
 Optionally, as second parameter available in both modalities, an integer can be provided that overrides the default period used to control the mouth, expressed in \mbox{[}ms\mbox{]}. Negative values are not processed and serve as placeholders. ~\newline
 Finally, available only in string mode, a third double can be provided that establishes the uttering duration in seconds, irrespective of the words actually spoken.
\item {\itshape /} $<$name$>$/emotions\+:o\+: this port serves to command the facial expressions.
\item {\itshape /} $<$name$>$/rpc\+: a remote procedure call port used for the following run-\/time querie\+: ~\newline

\begin{DoxyItemize}
\item \mbox{[}stat\mbox{]}\+: returns \char`\"{}speaking\char`\"{} or \char`\"{}quiet\char`\"{}.
\item \mbox{[}set\mbox{]} \mbox{[}opt\mbox{]} \char`\"{}package\+\_\+options\char`\"{}\+: set new package dependent command-\/line options.
\item \mbox{[}get\mbox{]} \mbox{[}opt\mbox{]}\+: returns a string containing the current package dependent command-\/line options.
\end{DoxyItemize}
\end{DoxyItemize}\hypertarget{group__windows-tts_tested_os_sec}{}\subsection{Tested O\+S}\label{group__windows-tts_tested_os_sec}
Linux and Windows.

\begin{DoxyAuthor}{Author}
Ugo Pattacini 
\end{DoxyAuthor}
